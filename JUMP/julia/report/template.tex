% keisoku_report.tex
\documentclass[dvipdfmx]{jsarticle}
\usepackage{listings,jlisting}  % ソースコードを行番号付きで載せるパッケージ
\usepackage[dvipdfmx]{color}              % 色付けするパッケージ
\usepackage[dvipdfmx]{graphicx}           % 画像を張り込むパッケージ
\usepackage{csvsimple}
\usepackage{subfigure}
\usepackage{amsmath}
\usepackage{mathtools}
\usepackage{url}
\usepackage{pdfpages}
%\usepackage{secdot}
%\usepackage{subcaption}

\setlength{\topmargin}{0mm}
\setlength{\oddsidemargin}{0mm}
\setlength{\textwidth}{16cm}
\setlength{\textheight}{23cm}

%\sectiondot{subsection}

\usepackage{ascmac}

\lstdefinelanguage{Julia}%
  {morekeywords={abstract,break,case,catch,const,continue,do,else,elseif,%
      end,export,false,for,function,immutable,import,importall,if,in,%
      macro,module,otherwise,quote,return,switch,true,try,type,typealias,%
      using,while},%
   sensitive=true,%
   alsoother={$},%
   morecomment=[l]\#,%
   morecomment=[n]{\#=}{=\#},%
   morestring=[s]{"}{"},%
   morestring=[m]{'}{'},%
}[keywords,comments,strings]%

%\renewcommand{\lstlistingname}{リスト} % 「ソースコード」を変更する
\lstset{language=C,% ソースの種類の指定
        basicstyle=\footnotesize,% リスト全体の設定
        commentstyle=\color{blue}\textit,% コメント部分の設定
        keywordstyle=\textbf,%  C言語の予約語(if,for,while等)の設定
        %keywordstyle=\color{red}\bfseries,%
        classoffset=1,%
        breakindent=20pt,%    改行時インデント量。デフォルト:20pt。
        breaklines=true,%   行が長くなってしまった場合の改行。
        frame=tlRB,framesep=7pt,% frame は top,left,right,bottom の1文字で指定、大文字は二重線
        showstringspaces=false,% string 中のスペースを記号表示するか
        numbers=left,% 行番号を付ける位置
        %stepnumber=2,%  何行ごとに行番号を表示するか デフォルトは 1
        numberstyle=\scriptsize\color{blue}% 行番号の表示スタイル
        }%

\begin{document}

\makeatletter % プリアンブルで定義開始

% 表示文字列を"図"から"Figure"へ
%\renewcommand{\figurename}{Figure}

% 図番号を"<章番号>.<図番号>" へ
\renewcommand{\thefigure}{\thesection.\arabic{figure}}

% 章が進むごとに図番号をリセットする
\@addtoreset{figure}{section}

\makeatother % プリアンブルで定義終了

\centerline{
    \huge
    ミニレポート第二回課題
}
\rightline{C0SB2018 小栗秀之}

リスト\ref{lst:cargo}に設定ファイル、リスト\ref{lst:rust}にソースコードを示す。

\begin{lstlisting}[caption=Cargo.toml, label=lst:cargo, language=C]
    [package]
    name = "int_rust"
    version = "0.1.0"
    edition = "2021"
    
    [dependencies]
    rand = "0.8.4"
\end{lstlisting}

\begin{lstlisting}[caption=mian.rs, label=lst:rust, language=C]
    use rand::{thread_rng, Rng};

    fn f(x: [f32; 10]) -> f32 {
        return x.iter().sum::<f32>().powf(2.0);
    }
    
    fn f_int( a: [(f32, f32); 10], n: i32) -> f32 {
    
        let mut rng = thread_rng();
    
        let mut sum: f32 = 0.0;
        for _t in 0..n {
                
            let mut x: [f32; 10] = [0.0; 10];
            for i in 0..10 {
                x[i] = rng.gen_range((a[i].0)..(a[i].1));
            }
    
            sum += f(x);
        }
        return a.iter().fold(1.0, |y, x| y*(x.1-x.0))*sum/(n as f32);
    }
    
    fn main() {
        
        let result: f32 = f_int(
            [
                (0.0, 1.0),
                (0.0, 1.0),
                (0.0, 1.0),
                (0.0, 1.0),
                (0.0, 1.0),
                (0.0, 1.0),
                (0.0, 1.0),
                (0.0, 1.0),
                (0.0, 1.0),
                (0.0, 1.0)
            ],
            10000000
        );
    
        println!("{}", result);
    }    
\end{lstlisting}


リスト\ref{lst:rust}に示したプログラムを実行した結果、積分の値は$25.821278$と計算できた。

\end{document}